%%%%%%%%%%%%%%%%%%%%%%%%%%%%%%%%%%%%%%%%%%%%%%%%%%%%%%%%%%%%%%%%%%%%%%%%%%%%%%%%
\documentclass[12pt,conference]{ieeeconf} %Github
%\documentclass[letterpaper, 12 pt, onecolumn]{ieeeconf} %Prof. Parallel

% Comment this line out
                                                          % if you need a4paper
%\documentclass[a4paper, 10pt, conference]{ieeeconf}      % Use this line for a4
                                                          % paper

\IEEEoverridecommandlockouts                              % This command is only
                                                          % needed if you want to
                                                          % use the \thanks command
\overrideIEEEmargins
% See the \addtolength command later in the file to balance the column lengths
% on the last page of the document

% The following packages can be found on http:\\www.ctan.org
\usepackage{graphics} % for pdf, bitmapped graphics files
\usepackage{epsfig} % for postscript graphics files
%\usepackage{mathptmx} % assumes new font selection scheme installed
%\usepackage{times} % assumes new font selection scheme installed
\usepackage{amsmath} % assumes amsmath package installed
\usepackage{amssymb}  % assumes amsmath package installed

\usepackage{tikz}
\usetikzlibrary{shapes, arrows.meta, positioning}

\usepackage{url}
\usepackage[ruled, vlined, linesnumbered]{algorithm2e}
%\usepackage{algorithm}
\usepackage{verbatim} 
%\usepackage[noend]{algpseudocode}
\usepackage{soul, color}
\usepackage{lmodern}
\usepackage[hidelinks]{hyperref}
\usepackage{fancyhdr}
\usepackage[utf8]{inputenc}
\usepackage{fourier} 
\usepackage{array}
\usepackage{pgf}
\usepackage{makecell}
\usepackage[sorting=none]{biblatex} % For biblatex
\addbibresource{reference.bib} % Path to your .bib file

\SetNlSty{large}{}{:}

\renewcommand\theadalign{bc}
\renewcommand\theadfont{\bfseries}
\renewcommand\theadgape{\Gape[4pt]}
\renewcommand\cellgape{\Gape[4pt]}

\newcommand{\rework}[1]{\todo[color=yellow,inline]{#1}}

\makeatletter
\newcommand{\rom}[1]{\romannumeral #1}
\newcommand{\Rom}[1]{\expandafter\@slowromancap\romannumeral #1@}
\makeatother

\pagestyle{plain} 

\title{GATto: topological information to improve node classification via GAT\\
\large Proposal for Learning from Network's project \\}

\author{Francesco Biscaccia Carrara, Alessandro Viespoli, Riccardo Modolo % <-this % stops a space 
\\\\ Master Degree in Computer Engineering \\
University of Padova \\
}

\begin{document}

\maketitle
\thispagestyle{plain}
\pagestyle{plain}

\section{MOTIVATION} 

Node classification is an important topic of graph analysis for find useful informations.\par
Quality of classification is an hot topic, since better classification present an advantage on extract conclusion or correlation in a graph.\par
Our intention is try to improve node classification using some precomputed features obtained via graph embedding and clustering.
To evaluate our hypothesis we use social graph with ground truth.
This type of graph help us to examine any difference between simple GAT and our GATto (GAT with topological features on each node) and check which approximate better the ground truth. \par

Time permitting and iif the method seem to work well, we put more effort in parllelization and scaling the framework to use it in a distributed environment.

%%%%%%%%%%%%%%%%%%%%%%%%%%%%%%%%%%%%%%%%%%%%%%%%%%%%%%%%%%%%%%%%%%%%%%%%%%%%%%%%
\section{METHODS}

We divide our workload in two main components
\begin{itemize}
    \item{\textbf{Precomputation module}: the part of code used to precompute additional features}
    \item{\textbf{GAT module}: the effective computation of node classification using a Graph Attention Network }
\end{itemize}

In the precomputation module we want to test different embedding library (node2vec or GraphWave) and different clustering parameters as features.\par
In our expectation we use the majority work effort in build the Precomputation module and conduct tests, since we want to use and already implemented GAT library.
We want also to be sure that our additional feature don't introduce any bias*** on the GAT network. 

%%%%%%%%%%%%%%%%%%%%%%%%%%%%%%%%%%%%%%%%%%%%%%%%%%%%%%%%%%%%%%%%%%%%%%%%%%%%%%%%

\section{Intended experiments}
We want to implement the methods in Python for the advantage of using deep learning library like tensorflow.
The code and all implementation details will be available on \textbf{GitHub}. \par

The intentional experiments approximately work as follow:
\begin{itemize}
    \item train GAT with default node features
    \item compute the classification error
    \item compute graph embedding on the graph
    \item compute clustering and add features to nodes
    \item train GAT with additional features
    \item compute the classification error
\end{itemize}

All tests were performed on the \textbf{\href{https://capri.dei.unipd.it}{CAPRI}} High-Performance Computing (HPC) system, owned by the University of Padova. CAPRI is designed to provide computational power for testing innovative algorithms across various research fields. It is equipped with the following hardware:
\begin{itemize}
    \item 16 Intel(R) Xeon(R) Gold 6130 @ 2.10GHz CPUs
    \item 6 TB of RAM
    \item 2 NVIDIA Tesla P100 16GB GPUs
    \item 40 TB of disk space
\end{itemize}

The objective of these tests is to observe if this precomputation provide any improvement on classification.\par
%\addtolength{\textheight}{-5cm}   % This command serves to balance the column lengths
                                  % on the last page of the document manually. It shortens
                                  % the textheight of the last page by a suitable amount.
                                  % This command does not take effect until the next page
                                  % so it should come on the page before the last. Make
                                  % sure that you do not shorten the textheight too much.

%%%%%%%%%%%%%%%%%%%%%%%%%%%%%%%%%%%%%%%%%%%%%%%%%%%%%%%%%%%%%%%%%%%%%%%%%%%%%%%%

%%%%%%%%%%%%%%%%%%%%%%%%%%%%%%%%%%%%%%%%%%%%%%%%%%%%%%%%%%%%%%%%%%%%%%%%%%%%%%%%
\printbibliography[nottype=online]
\end{document}