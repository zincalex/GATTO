%%%%%%%%%%%%%%%%%%%%%%%%%%%%%%%%%%%%%%%%%%%%%%%%%%%%%%%%%%%%%%%%%%%%%%%%%%%%%%%%
\documentclass[12pt,conference]{ieeeconf} %Github
%\documentclass[letterpaper, 12 pt, onecolumn]{ieeeconf} %Prof. Parallel

% Comment this line out
                                                          % if you need a4paper
%\documentclass[a4paper, 10pt, conference]{ieeeconf}      % Use this line for a4
                                                          % paper

\IEEEoverridecommandlockouts                              % This command is only
                                                          % needed if you want to
                                                          % use the \thanks command
\overrideIEEEmargins
% See the \addtolength command later in the file to balance the column lengths
% on the last page of the document

% The following packages can be found on http:\\www.ctan.org
\usepackage{graphics} % for pdf, bitmapped graphics files
\usepackage{epsfig} % for postscript graphics files
%\usepackage{mathptmx} % assumes new font selection scheme installed
%\usepackage{times} % assumes new font selection scheme installed
\usepackage{amsmath} % assumes amsmath package installed
\usepackage{amssymb}  % assumes amsmath package installed

\usepackage{tikz}
\usetikzlibrary{shapes, arrows.meta, positioning}

\usepackage{url}
\usepackage[ruled, vlined, linesnumbered]{algorithm2e}
%\usepackage{algorithm}
\usepackage{verbatim} 
%\usepackage[noend]{algpseudocode}
\usepackage{soul, color}
\usepackage{lmodern}
\usepackage[hidelinks]{hyperref}
\usepackage{fancyhdr}
\usepackage[utf8]{inputenc}
\usepackage{fourier} 
\usepackage{array}
\usepackage{pgf}
\usepackage{makecell}
\usepackage[sorting=none]{biblatex} % For biblatex
\addbibresource{reference.bib} % Path to your .bib file

\SetNlSty{large}{}{:}

\renewcommand\theadalign{bc}
\renewcommand\theadfont{\bfseries}
\renewcommand\theadgape{\Gape[4pt]}
\renewcommand\cellgape{\Gape[4pt]}

\newcommand{\rework}[1]{\todo[color=yellow,inline]{#1}}

\makeatletter
\newcommand{\rom}[1]{\romannumeral #1}
\newcommand{\Rom}[1]{\expandafter\@slowromancap\romannumeral #1@}
\makeatother

\pagestyle{plain} 

\title{GATTO: Can Topological Information Improve Node Classification via GAT?\\
\large Proposal for Learning from Network's project \\}

\author{Francesco Biscaccia Carrara \textit{(2120934)}, Riccardo Modolo \textit{(2123750)},\\ Alessandro Viespoli \textit{(2120824)} % <-this % stops a space 
\\\\ Master Degree in Computer Engineering \\
University of Padova \\
}

\begin{document}

\maketitle
\thispagestyle{plain}
\pagestyle{plain}

%%%%%%%%%%%%%%%%%%%%%%%%%%%%%%%%%%%%%%%%%%%%%%%%%%%%%%%%%%%%%%%%%%%%%%%%%%%%%%%%
\section{MOTIVATION} 
Node classification is an important topic in graph analysis for assigning to each node a label from a set of predefined classes. \\
Our intention is to see whether pre-computing features obtained via graph embedding and clustering can improve node classification via graph attention networks (GAT)$^\text{\cite{GAT}}$. 

\section{DATA}
The datasets we will be using are taken from Stanford Network Analysis Project (SNAP)$^\text{\cite{SNAP}}$: 
\begin{table}[h!]
\centering
\renewcommand{\arraystretch}{1.5}
\begin{tabular}{|l|c|c|c|}
\hline
\textbf{Network}           & \textbf{Nodes} & \textbf{Edges} & \textbf{Communities} \\
\hline
email-EU-core  & 1005           & 25571          & 42         \\
com-Amazon     & 334863         & 925872         & 75149    \\
wiki-topcats        & 1791489        & 28511807       & 17364    \\
\hline
\end{tabular}
\end{table}
%%%%%%%%%%%%%%%%%%%%%%%%%%%%%%%%%%%%%%%%%%%%%%%%%%%%%%%%%%%%%%%%%%%%%%%%%%%%%%%%
\section{METHODS}
In order to evaluate our hypothesis we will use social graph with ground truth, in this way we will examine any difference between simple GAT classification and our GATTO (GAT with topological features on each node) structure. Our hope is to improve classification performances with simple and fast precomputation on the graph to gather useful features which will be used by the GAT. \\
We divide our workload in two main components:
\begin{itemize}
    \item\textbf{Precomputation module}: the part of code used to precompute additional features.
    \item\textbf{GAT module}: the effective computation of node classification using a GAT. 
\end{itemize}
%In the precomputation module we want to test different embedding library (i.e. node2vec$^\text{\cite{node2vec}}$) and then from the prev calculate different clustering parameters like:
In the precomputation module we want to test an embedding library (i.e. node2vec$^\text{\cite{node2vec}}$) and then from the embedding calculate different clustering parameters like:
\begin{itemize}
    \item the cluster of the node
    \item its nearest cluster
    \item the distance of a node to the center of its cluster
    \item the distance of a node to its nearest cluster's center
\end{itemize}
%The majority of the work will be building the precomputation module and conduct tests, since we will be using an already implemented GAT library.

%%%%%%%%%%%%%%%%%%%%%%%%%%%%%%%%%%%%%%%%%%%%%%%%%%%%%%%%%%%%%%%%%%%%%%%%%%%%%%%%

\section{INTENDED EXPERIMENTS}
We want to implement the overall architecture in Python, leveraging the available algorithms' implementation: 
\begin{enumerate}
    \item \textit{node2vec}$^\text{\cite{n2vGitHub}}$ for node embedding task;
    \item \textit{scikit-learn}$^\text{\cite{scikitGitHub}}$ for clustering task;
    \item \textit{spektral}$^\text{\cite{spektralGitHub}}$ for GAT implementation.
\end{enumerate}
The intentional experiments will be approximately:
\begin{itemize}
    \item train GAT with default node features in the dataset.
    \item compute the classification error.
    \item compute additional graph features by using algorithms within the precomputation module.
    \item train GAT with the new additional features.
    \item compute the classification error.
    \item conduct analytical tests. 
\end{itemize}
The code and all implementation details will be available on GitHub. \par
%\href{https://capri.dei.unipd.it}
All tests will be performed on the \textbf{CAPRI}$^\text{\cite{CAPRI}}$ High-Performance Computing (HPC) system, owned by the University of Padova. It is equipped with the following hardware:
\begin{itemize}
    \item 16 Intel(R) Xeon(R) Gold 6130 @ 2.10GHz CPUs
    \item 6 TB of RAM
    \item 2 NVIDIA Tesla P100 16GB GPUs
    \item 40 TB of disk space
\end{itemize}
Time permitting and if and only if the method seems to achieve some results, we will put more effort in parallelization and scaling the framework to a distributed environment.

%%%%%%%%%%%%%%%%%%%%%%%%%%%%%%%%%%%%%%%%%%%%%%%%%%%%%%%%%%%%%%%%%%%%%%%%%%%%%%%%
\vspace{\fill}
\printbibliography
%%%%%%%%%%%%%%%%%%%%%%%%%%%%%%%%%%%%%%%%%%%%%%%%%%%%%%%%%%%%%%%%%%%%%%%%%%%%%%%%
% WHAT MIGHT BE ADDED IN THE NEXT UPDATE PROPOSAL 
% --> list of analytical test to be conducted 
% --> in case the fact that the GAT cannot compute directed graphs
% --> more local graph features that came up to our mind
% --> everything that will be modified (DAHHH)
\end{document}