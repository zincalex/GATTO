%%%%%%%%%%%%%%%%%%%%%%%%%%%%%%%%%%%%%%%%%%%%%%%%%%%%%%%%%%%%%%%%%%%%%%%%%%%%%%%%
\documentclass[12pt,conference]{ieeeconf} %Github
%\documentclass[letterpaper, 12 pt, onecolumn]{ieeeconf} %Prof. Parallel

% Comment this line out
                                                          % if you need a4paper
%\documentclass[a4paper, 10pt, conference]{ieeeconf}      % Use this line for a4
                                                          % paper

\IEEEoverridecommandlockouts                              % This command is only
                                                          % needed if you want to
                                                          % use the \thanks command
\overrideIEEEmargins
% See the \addtolength command later in the file to balance the column lengths
% on the last page of the document

% The following packages can be found on http:\\www.ctan.org
\usepackage{graphics} % for pdf, bitmapped graphics files
\usepackage{epsfig} % for postscript graphics files
%\usepackage{mathptmx} % assumes new font selection scheme installed
%\usepackage{times} % assumes new font selection scheme installed
\usepackage{amsmath} % assumes amsmath package installed
\usepackage{amssymb}  % assumes amsmath package installed

\usepackage{tikz}
\usetikzlibrary{shapes, arrows.meta, positioning}

\usepackage{url}
\usepackage[ruled, vlined, linesnumbered]{algorithm2e}
%\usepackage{algorithm}
\usepackage{verbatim} 
%\usepackage[noend]{algpseudocode}
\usepackage{soul, color}
\usepackage{lmodern}
\usepackage[hidelinks]{hyperref}
\usepackage{fancyhdr}
\usepackage[utf8]{inputenc}
\usepackage{fourier} 
\usepackage{array}
\usepackage{pgf}
\usepackage{makecell}
\usepackage[sorting=none]{biblatex} % For biblatex
\addbibresource{reference.bib} % Path to your .bib file

\SetNlSty{large}{}{:}

\renewcommand\theadalign{bc}
\renewcommand\theadfont{\bfseries}
\renewcommand\theadgape{\Gape[4pt]}
\renewcommand\cellgape{\Gape[4pt]}

\newcommand{\rework}[1]{\todo[color=yellow,inline]{#1}}

\makeatletter
\newcommand{\rom}[1]{\romannumeral #1}
\newcommand{\Rom}[1]{\expandafter\@slowromancap\romannumeral #1@}
\makeatother

\pagestyle{plain} 

\title{GATTO: Can Topological Information Improve Node Classification via GAT?\\
\large Midterm Report for Learning from Network's project \\}

\author{Francesco Biscaccia Carrara \textit{(2120934)}, Riccardo Modolo \textit{(2123750)},\\ Alessandro Viespoli \textit{(2120824)} % <-this % stops a space 
\\\\ Master Degree in Computer Engineering \\
University of Padova \\
}

\begin{document}

\maketitle
\thispagestyle{plain}
\pagestyle{plain}

%%%%%%%%%%%%%%%%%%%%%%%%%%%%%%%%%%%%%%%%%%%%%%%%%%%%%%%%%%%%%%%%%%%%%%%%%%%%%%%%
\section{INTRODUCTION} 
The scope of this document is to:
\begin{itemize}
    \item Clarify all points not properly explained in the proposal paper
    \item Summarize all work done
    \item Define the missing work 
    \item Estimate time and effort to finish the project.
\end{itemize}

TODO: inserire riassuntino di quello che scriviamo dopo

\section{CLARIFICATION POINTS}
One of the most important clarification point is about data: we didn't provide an extensive explanation, in the proposal paper, about how data are built and how we want to use them.
The graphs we choose in SNAP$^\text{\cite{SNAP}}$ are undirected. Each node contain community as parameter. 

\section{WORK DONE}
We've already build the \textbf{precomputation module} (explained in the project proposal) 
and the code to automate the testing phase inside cluster. We've decided the hyperparameters
about node2vec.

\section{WORK IN PROGRESS}
We nedd to build the \textbf{GAT module} (Viespoli) and do the tests on \textbf{CAPRI}$^\text{\cite{CAPRI}}$.
Also writing the paper (for obvious reason) is a missing step. 

\section{ESTIMATION}
Since we need our collegue (Viespoli) to build the GAT, we estimate to finish the project on 7$^{\text{th}}$ January.
With this estimation we have 8 days for eventual issue and refinements.
%%%%%%%%%%%%%%%%%%%%%%%%%%%%%%%%%%%%%%%%%%%%%%%%%%%%%%%%%%%%%%%%%%%%%%%%%%%%%%%%
\vspace{\fill}
\printbibliography
%%%%%%%%%%%%%%%%%%%%%%%%%%%%%%%%%%%%%%%%%%%%%%%%%%%%%%%%%%%%%%%%%%%%%%%%%%%%%%%%
% WHAT MIGHT BE ADDED IN THE NEXT UPDATE PROPOSAL 
% --> list of analytical test to be conducted 
% --> in case the fact that the GAT cannot compute directed graphs
% --> more local graph features that came up to our mind
% --> everything that will be modified (DAHHH)
\end{document}