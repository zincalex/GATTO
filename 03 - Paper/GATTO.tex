%%%%%%%%%%%%%%%%%%%%%%%%%%%%%%%%%%%%%%%%%%%%%%%%%%%%%%%%%%%%%%%%%%%%%%%%%%%%%%%%
\documentclass[12pt,conference]{ieeeconf} %Github
%\documentclass[letterpaper, 12 pt, onecolumn]{ieeeconf} %Prof. Parallel

% Comment this line out
                                                          % if you need a4paper
%\documentclass[a4paper, 10pt, conference]{ieeeconf}      % Use this line for a4
                                                          % paper

\IEEEoverridecommandlockouts                              % This command is only
                                                          % needed if you want to
                                                          % use the \thanks command
\overrideIEEEmargins
% See the \addtolength command later in the file to balance the column lengths
% on the last page of the document

% The following packages can be found on http:\\www.ctan.org
\usepackage{graphics} % for pdf, bitmapped graphics files
\usepackage{epsfig} % for postscript graphics files
%\usepackage{mathptmx} % assumes new font selection scheme installed
%\usepackage{times} % assumes new font selection scheme installed
\usepackage{amsmath} % assumes amsmath package installed
\usepackage{amssymb}  % assumes amsmath package installed

\usepackage{tikz}
\usetikzlibrary{shapes, arrows.meta, positioning}

\usepackage{url}
\usepackage[ruled, vlined, linesnumbered]{algorithm2e}
%\usepackage{algorithm}
\usepackage{verbatim} 
%\usepackage[noend]{algpseudocode}
\usepackage{soul, color}
\usepackage{lmodern}
\usepackage[hidelinks]{hyperref}
\usepackage{fancyhdr}
\usepackage[utf8]{inputenc}
\usepackage{fourier} 
\usepackage{array}
\usepackage{pgf}
\usepackage{makecell}
\usepackage[sorting=none]{biblatex} % For biblatex
\addbibresource{reference.bib} % Path to your .bib file

\SetNlSty{large}{}{:}

\renewcommand\theadalign{bc}
\renewcommand\theadfont{\bfseries}
\renewcommand\theadgape{\Gape[4pt]}
\renewcommand\cellgape{\Gape[4pt]}

\newcommand{\rework}[1]{\todo[color=yellow,inline]{#1}}

\makeatletter
\newcommand{\rom}[1]{\romannumeral #1}
\newcommand{\Rom}[1]{\expandafter\@slowromancap\romannumeral #1@}
\makeatother

\pagestyle{plain} 

\title{GATTO: Can Topological Information Improve Node Classification via GAT?\\
\large Final Report for Learning from Network's project \\}

\author{Francesco Biscaccia Carrara \textit{(2120934)}, Riccardo Modolo \textit{(2123750)},\\ Alessandro Viespoli \textit{(2120824)} % <-this % stops a space 
\\\\ Master Degree in Computer Engineering \\
University of Padova \\
}

\begin{document}

\maketitle
\thispagestyle{plain}
\pagestyle{plain}

%%%%%%%%%%%%%%%%%%%%%%%%%%%%%%%%%%%%%%%%%%%%%%%%%%%%%%%%%%%%%%%%%%%%%%%%%%%%%%%%
\section{ABSTRACT} 
In this report we study how topological features of a node can affect its prediction on GAT (Graph Attention Network).
We introduce the dataset, the type of feature we computed and we compare our result with the 
original GAT paper. We also use a graph with ground truth but without freatures to observe how only 
topological feature produce prediction. 

%%%%%%%%%%%%%%%%%%%%%%%%%%%%%%%%%%%%%%%%%%%%%%%%%%%%%%%%%%%%%%%%%%%%%%%%%%%%%%%%
\section{INTRODUCTION} 
Node classification is an important reasearch and business topic.
Our framework GATTO (Graph ATtention network with TOpological information) want to 
improve the classic GAT prediction using node feature coming from the graph or from its 
embedding. The main idea is to using topological features to imporve prediction of GAT, where each node already have features in it. 
Another possible use is when node didn't have any features, so we compute and use them for GAT train.
In this paper we explore both scenarios. 

%%%%%%%%%%%%%%%%%%%%%%%%%%%%%%%%%%%%%%%%%%%%%%%%%%%%%%%%%%%%%%%%%%%%%%%%%%%%%%%%
\section{DATA} 
We're going to use the same dataset of GAT$^\text{\cite{GAT}}$, with features to each node, and one Graph with ground truth from SNAP $^\text{\cite{SNAP}}$ (without features).
the Dataset of GAT are: \textit{Cora, Citeseer and Pubmed}. For SNAP, we have \textit{email-Eu-core}. Each node have only one label and the graph is mix directed/undirected
\\
\begin{table}[h!]
    \centering
    \renewcommand{\arraystretch}{1.5}
    \begin{tabular}{|l|c|c|c|c|}
    \hline
    \textbf{Network} & \textbf{Nodes} & \textbf{Edges} & \textbf{labels} & \textbf{features} \\
    \hline
    email-EU-core  & 1005 & 25571 & 42 & 0     \\
    Cora           & 2708 & 5429  & 7  & 1443  \\
    Citeseer       & 3327 & 4732  & 6  & 3703  \\
    Pubmed        & 19717 & 44338 & 3  & 500   \\
    \hline
    \end{tabular}
\end{table}
\\
The feature we intended to compute and assign to each node for all these graphs are:
\begin{itemize}
    \item degree centrality
    \item betweenness centrality
    \item closeness centrality
    \item suggested label
\end{itemize}
The \textit{suggested label} parameter is the result of a clustering made on the embedding of the Graph.
We use Node2Vec$^\text{\cite{node2vec}}$ to produce the embedding, and for clustering we use k-mean++ method.


%%%%%%%%%%%%%%%%%%%%%%%%%%%%%%%%%%%%%%%%%%%%%%%%%%%%%%%%%%%%%%%%%%%%%%%%%%%%%%%%
\section{IMPLEMENTATION} 
The practical implementation$^{\text{\cite{scikitGitHub}}}$ respect the nature of the the framework concept.
We have two block:
\begin{itemize}
    \item{\textbf{Precomputation Module}: the class that compute every needed or requested features from the graph or from it's embedding, and return it as feature matrix}
    \item{\textbf{GAT Module}: the GAT implementation for train and predict node labels}
\end{itemize}


%%%%%%%%%%%%%%%%%%%%%%%%%%%%%%%%%%%%%%%%%%%%%%%%%%%%%%%%%%%%%%%%%%%%%%%%%%%%%%%%
\section{TESTS} 

%%%%%%%%%%%%%%%%%%%%%%%%%%%%%%%%%%%%%%%%%%%%%%%%%%%%%%%%%%%%%%%%%%%%%%%%%%%%%%%%
\section{RESULTS} 

%%%%%%%%%%%%%%%%%%%%%%%%%%%%%%%%%%%%%%%%%%%%%%%%%%%%%%%%%%%%%%%%%%%%%%%%%%%%%%%%
\vspace{\fill}
\printbibliography
\newpage
\section*{Work Report}
In this section we want to describe the distribution of the work and detailed contribution of each member.
\begin{itemize}
    \item \textit{Francesco Biscaccia Carrara (2120934)}: {\textbf{33.3\% of the work}. Produce the code compute features and Runnig test on CAPRI}\\
    \item \textit{Alessandro Viespoli (2120824)}: {\textbf{33.3\% of the work}. Produce the code for the GAT and the code for plotting results}\\
    \item \textit{Riccardo Modolo (2123750)}: {\textbf{33.3\% of the work}. Produce the code to retrieve data, review code, write Proposal, Midterm  and final Paper}
\end{itemize}
%%%%%%%%%%%%%%%%%%%%%%%%%%%%%%%%%%%%%%%%%%%%%%%%%%%%%%%%%%%%%%%%%%%%%%%%%%%%%%%%
\end{document}